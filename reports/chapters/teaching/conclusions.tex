\section{Conclusions}
In the chapter, two OERs focussed on classical simulation and MD were shown.
The first of these resources was the \texttt{pylj} Python package, which designed for use at any education level to give an easy, visual example of classical simulation.
This software is open-source and actively developed, with the growth of capability and applications in the future.
Currently, the code is used in the third year computational chemistry laboratory at the University of Bath and there is an ongoing discussion for it to be used in future in the second year computational chemistry laboratory at the University of Bristol.
Additionally, the webpage for \texttt{pylj} at \href{https://pythoninchemistry.org/pylj}{pythoninchemistry.org/pylj} has been viewed over 400 times since launching in June 2018m indicating the popularity of the software.

The second OER is the online, interactive learning module for the introduction of users of experimental scattering methods to classical simulation.
This module is shared under an open, permissive license and in future I hope that its use/reuse will be uptaken by educators of scattering worldwide.
Furthermore, there is scope to introduce the module as a flipped learning component\autocite{noauthor_flipped_nodate} within scattering courses at the ISIS Neutron and Muon Source and Diamond Light Source.

While the \texttt{pylj} software improves the availability of resources to introduce MD simulation, this interactive learning module represents a unique resource to enable learning and understanding.
Previously, MD simulation has been taught in a chalk-and-talk fashion, or with examples of working simulations, however, this resources enables students to learn about the simulation methodology by interacting directly with working code.
This allows of the worked example effect to be envoked during learning, improving the educational power of the materials.
