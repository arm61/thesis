\section{Probing radiation}

This work is focussed on the use of X-ray and neutron scattering to probe soft matter systems; in particular surfactant monolayers and micelles. Therefore, it is pertinent to discuss how each of these probing radiation is produced and detail the advantages of each with respect to the other.

\subsection{Generation of X-rays}

X-rays are a form of electromagnetic radiation similar to visible light, albeit with a much shorter wavelength, from \SI{0.01}{\nano\meter} to \SI{10}{\nano\meter}.
There are four common ways to produce X-rays; three are available within the laboratory, while the other is exclusive to large scale facilities.

The three laboratory source X-ray generation techniques are the X-ray tube, the rotating anode, and the liquid jet.
An X-ray tube consists of a filament and an anode within a vacuum chamber, by passing a high voltage electrical current across the filament electrons are emitted which accelerate towards the anode.
On collision with the anode, the rapid deceleration results in the emission of X-rays of a characteristic wavelength based on the anode material \cite{schnablegger_saxs_2017}.
The most common material for an X-ray tube anode is copper which gives off radiation of about \SI{8}{\kilo\eV}.

Another common laboratory method for the generation of X-rays is the rotating anode, which is an improvement on the X-ray tube.
In the X-ray tube, each time that an electron contacts the anode there is some energy transfer, this means that over many millions of collisions, the temperature of the anode can rise significantly, leading to a temperature limitation, due to the melting of the anode material, on the X-ray flux available.
This lead to the development of the rotating anode, this is simply where the anode is made from a rotating wheel, so that the bombardment is spread across the whole wheel reducing the energy localisation.
This allows an increase in the photon flux by about an order of magnitude \cite{schnablegger_saxs_2017}.

The final laboratory method for X-ray generation is the liquid jet source (branded MetalJet by excillum) \cite{noauthor_metaljet_nodate}.
For the liquid jet X-ray source, an electron beam is incident on a liquid metal sample (usually gallium or indium alloy), rather than traditional solid metal.
This means that the electron intensity and therefore X-ray brightness, available to the liquid jet source is much greater than a rotating anode source.

The final method of X-ray generation is at a synchrotron facility, this method has the drawback that it requires access to a national or international facility; such as Diamond Light Source (DLS) or the European Synchrotron Radiation Facility (ESRF).
The way in which X-rays are generated at the synchrotron involves the acceleration of an electron, rather than the deceleration as with the laboratory sources.
This is achieved by having relativistic electrons travel around a curve, from Newtonian mechanics it is known that travelling on a curve at constant speed is equivalent to acceleration.
First electrons are accelerated, after being produced in a linear accelerator (Linac), to near the speed of light in a booster synchrotron before injecting them into the storage ring.
In the storage ring, the electrons are kept at relativistic speeds with bending magnets (BM) and straight sections making up a ring (Figure~\ref{fig:syn}).
The circularity of the ring is dependent on the number of bending magnets that make up the ring; for example, DLS had 48 bending magnets with 48 straight sections at the time of construction.
%
\begin{figure}
    \centering
    \includegraphics[width=0.85\textwidth]{theory/syn}
    \caption{A schematic representation of a synchrotron radiation source, identifying the Linac, the booster ring, the radio-frequency cavities (rf), the bending magnet (BM) and the insertion device (ID), from Reference~\cite{garcia-gutierrez_bases_2009}. Reprinted/adapted by permission from Springer Nature Customer Service Centre GmbH: Springer Nature Bases of Synchrotron Radiation, Light Sources, and Features of X- Ray Scattering Beamlines by M.C. Garc\'{i}a-Guti\'{e}rrez, D.R. Rueda\textsuperscript{\textcopyright} (2009).}
    \label{fig:syn}
\end{figure}
%

When an electron accelerates (or travels on a curve), Cherenkov radiation is emitted in accordance with the Cherenkov relation,
%
\begin{equation}
    n_i\beta_c\cos{\theta_e} = 1,
\end{equation}
%
where, $n_i$ is the refractive index for the dielectric medium, $\beta_c$ is the fraction of the speed of light at which that electron is travelling, and $\theta_e$ is the angle between the electron trajectory and the trajectory of the resulting photon \cite{garcia-gutierrez_bases_2009}.
The curve is the result of a bending magnet, meaning that at each bending magnet there can be a beamline which gives out synchrotron light.
The light that is given off from a bending magnet is continuous and broad, covering a wide range of the electromagnetic spectrum.
The alternative to a bending magnet beamline is a beamline which is served by an insertion device (ID).
An insertion device is able to offer more specific radiation characteristics (photon energy, narrower band) than a bending magnet, and are placed on the straight sections of the synchrotron.
Common insertion devices include wavelength shifters, wigglers, and undulators.

The type of insertion device that is present at both I07 and I22 at DLS is an undulator.
An undulator consists of a series of magnets of opposing polarity which causes the electrons to `wiggle' back and forth (Figure~\ref{fig:undulator}).
This results in a superposition of radiation from $N_P$ sources, where $N_P$ is the number of magnets, yielding quasi-monochromatic radiation.
The brilliance of different X-ray sources are compared in Table \ref{tab:sources}, this shows the significant benefit that an undulator offers in terms of photon brilliance.
%
\begin{figure}
    \centering
    \includegraphics[width=0.85\textwidth]{theory/undulator}
    \caption{A diagram of an undulator insertion device such as that on I07 or I22 where $\lambda_P$ is the period length between opposing magnets, from Reference~\cite{garcia-gutierrez_bases_2009}. Reprinted/adapted by permission from Springer Nature Customer Service Centre GmbH: Springer Nature Bases of Synchrotron Radiation, Light Sources, and Features of X- Ray Scattering Beamlines by M.C. Garc\'{i}a-Guti\'{e}rrez, D.R. Rueda\textsuperscript{\textcopyright} (2009).}
    \label{fig:undulator}
\end{figure}
%
%
\begin{table}
    \centering
    \small
    \caption{A comparision of the photon brilliance from different light sources. Adapted, with permission of Oxford University Press\textsuperscript{\textcopyright}, from Reference~\cite{sivia_elementary_2011}.}
    \label{tab:sources}
    \begin{tabular}{l | c}
        \toprule
        \multirow{2}{*}{Light source } & Approximate brilliance/ \\
 & \si{photons\,\second^{-1}\milli\radian^{-2}{0.1}\percent bandwidth^{-1}} \\
        \midrule
        Candle & $10^5$ \\
        X-ray tube & $10^8$ \\
        Sun & $10^{10}$ \\
        Bending magnet & $10^{15}$ \\
        Undulator & $10^{20}$ \\
        \bottomrule
    \end{tabular}
\end{table}
%

\subsection{Generation of neutrons}
\label{sec:neutrons}

Neutrons hold an advantage over X-rays, particularly for application to the study of soft matter, in the ability to utilise contrast variation to increase the quantity of information from the sample, this is discussed in detail in Section~\ref{convar}.
However, neutrons cannot be produced safely on a laboratory scale, therefore it is always necessary to visit large scale facilities to harness neutrons for scattering experiments.
These facilities come in two flavours; the reactor source and the spallation source, each offering unique benefits.

Neutron reactor sources, such as the Institut Laue-Langevin (ILL) in Grenoble, France, are currently the most common format of neutron source and are capable of producing the highest average neutron flux, the number of neutrons per second per unit area, for example, the High-Flux Reactor at the ILL is capable of producing a neutron flux of \SI{1.5E15}{neutrons\,\second^{-1}\centi\meter^{-2}} \cite{noauthor_ill_nodate}.
A reactor source operates on the principle of nuclear fission, where an atomic nucleus is capable of breaking down into smaller nuclei, overcoming the strong nuclear force. This often involves using uranium enriched with its fissile isotope, \ce{^{235}U}, which after the initial absorption of a stray neutron, from a cosmic ray, or spontaneous fission, will undergo fission to release, on average, 2.5 daughter neutrons, an example of a possible uranium fission mechanism is:
%
\begin{equation*}
    \ce{n + ^{235}U -> ^{236}U -> ^{134}Xe + ^{100}Sr + 2n}.
\end{equation*}
%
This type of mechanism is the basis for research, and nuclear power, reactors \cite{sivia_elementary_2011}.
One of the major drawbacks for reactor neutron sources is the perceived public opinion towards such facilities.
Major safety concerns, such as ``nuclear meltdown'' and the resulting nuclear waste, mean that reactor sources are often unpopular and therefore struggle to obtain funding required for operation.

The other form of neutron source is a spallation source, this is much less controversial as it does not require fissile materials and hence there is no risk of a nuclear disaster.
The ISIS Neutron and Muon Source (Oxfordshire, UK) is an example of a spallation source, where high energy protons, \SI{800}{\mega\eV} \cite{noauthor_isis_nodate-1}, are accelerated towards a tungsten target.
When the protons strike the target, they can cause the release of a series of neutrons, the first batch of neutrons are given off with too high an energy to be useful, however, less excited neutrons are given off by secondary emissions.
In addition to the public perception benefit, spallation sources also have a technological advantage in the time-of-flight technique.
The time-of-flight (ToF) technique is based on the fact that at a spallation source, it is possible to know the time at which the neutron was ejected by the target to a high level of precision and therefore it is possible to measure the time taken for the neutron to reach the instrument.
Since the neutron is a particle of a finite mass, $m$, it is possible to correlate the velocity, $v$, of the particle with the kinetic energy, $E_k$,
%
\begin{equation}
    E_k = \frac{mv^2}{2},
\end{equation}
%
and with knowledge of the energy of the particle, its wavelength $\lambda$, can be determined by the de Broglie relation,
%
\begin{equation}
    E = h\omega = \frac{hv}{\lambda},
\end{equation}
%
where, $h$ is Planck's constant and $\omega$ is the neutron frequency.
Therefore, the wavelength of the neutron is proportional to the inverse of the particle's velocity, and hence the time-of-flight, $t_F$,
%
\begin{equation}
    \lambda = \frac{h}{mv} = \frac{ht_F}{mL_F},
\end{equation}
%
where, $L_F$ is the distance between the target and the instrument.
The fact that the neutrons can spread out in the flight from the target means that wavelength-dispersive techniques, where the neutron wavelength is measured rather than the scattering angle, are possible at spallation sources which cannot be carried out at reactor sources.
The weakness of current spallation sources is that they have a lower average flux than reactor sources, however, the building of the European Spallation Source (ESS) will change this as it offers an average flux similar to that of a reactor source, but with the benefits of the spallation technique.

A problem that is inherent for both reactor and spallation sources is that the energy of the neutrons given off is usually too high to be used to study condensed materials, such as soft matter.
This means that moderation must be used to reduce the energy of the neutrons passing through the sample.
The neutrons which are considered to be optimal for the study of condensed materials are thermal neutrons, named because their energy is approximately that of ambient temperature.
Thermal neutrons are achieved by allowing the neutrons to pass through a large volume of moderator material, usually, graphite heavy water (\ce{D2O}), methane or \ce{H2}, stored at \SI{300}{\kelvin} before they reach the instrument \cite{sivia_elementary_2011}.
