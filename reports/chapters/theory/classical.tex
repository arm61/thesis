\section{Classical simulation}
\label{sec:classical}

In order to simulation a real chemical system, it is necessary to model the electrons of the molecules and their interactions.
This is usually achieved using quantum mechanical calculations, where the energy of the system is calculated by finding som approximate solution to the Schr\"{o}dinger equation.
However, quantum mechanical calculations are very computationally expensive, and are realistically limited to hundreds of atoms.
In order to simulate a soft matter system such as a lipid monolayer or polymer nanoparticles, it is necessary to simply the calculation being performed.
This leads to classical simulation, where mathematical functions are used to determined the potential energy of the system.
Classical simulations is used substantially in this work, in terms of both molecular dynamics simulations and energy minimisation methods (see Section \ref{sec:simulation}).
Therefore, it is necessary to introduce the underlying theory on which this method is defined.

\subsection{Potential models}


\subsection{Parameterisation}
