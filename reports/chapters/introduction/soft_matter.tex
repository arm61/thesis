\section{Soft matter}
Soft matter is an umbrella term for many different types of material.
These include micelles; sub-micron sized, dynamic agglomerates of amphiphilic molecules such as surfactants or block co-polymers, colloidal solutions; where the interaction between the colloids may be controlled through chemical modification, or proteins; where the polar nature of different amino acids leads to the protein folding into a highly organised, and biologically relevant, shape.
Some examples of these soft matter systems are shown in Figure~\ref{fig:soft}.
These species, initially, appear rather disparate, however, there are a few important commonalities among soft matter systems \sidecite{jones_soft_2002}:
\begin{itemize}
  \item the lengths scales are intermediate between atomistic and macroscopic,\footnote{typically \SIrange{1e-8}{1e-5}{\meter}.}
  \item for soft matter systems the energy of a structural distortion is similar to thermal energy, so the material is in constant flux,
  \item this thermal motion can lead to the formation of complex, hierarchical structures due to the balance between enthalpy and entropy, this process is referred to as self-assembly.
\end{itemize}
%
\begin{figure}[t]
    \forceversofloat
    \centering
    \includegraphics[width=\textwidth]{introduction/soft_matter_examples}
    \caption{Three examples of soft matter species; (a) a 43 \ce{C_{10}TAB} surfactant micelle, reprinted with permission from \cite{hargreaves_atomistic_2011}, copyright 2011 American Chemical Society, (b) the tunable interactions of colloids, reprinted with permission from \cite{kraft_patchy_2011}, copyright 2011 American Chemical Society, and (c) generated using VMD (\cite{humphrey_vmd_1996}) from the crystal structure of T4-lysozyme \cite{rose_crystal_1988}.}
    \label{fig:soft}
\end{figure}
%

Soft matter self-assembly is the ability for soft matter systems to form organised structures in solution.
These are of particular interest industrially, where surfactant and polymer self-assembly play an import role in food, commodity, and speciality chemicals.\sidecite{schramm_surfactants_2003}
Self-assembly processes are important from a biological perspective as it is phospholipids, a family of surface-active biomolecules, which make up the bilayers that protect cells.\sidecite{simons_lipid_2000}
The structures that result from the self-assembly of soft matter species have fluid-like properties.
This is due to the fact that the subunits are held together by weak forces such as the van der Waals, hydrophobic, hydrogen-bonding, and screened electrostatic interactions.\sidecite{israelachvili_intermolecular_2011}
This means that the structure of a self-assembled species is susceptible to changes in the local chemical environment, such as pH or salt concentration.\sidecite{schmaljohann_thermo-_2006,sammalkorpi_ionic_2009}

The focus of this work is on the self-assembly of surfactant molecules.
Surfactant is a general term for any molecule which is ``surface-active'', to say that they will interact at an interface.\sidecite{rosen_surfactants_2012}
Surfactants are generally made up of two components; one part is highly soluble in one of the interfacial phases, while the other is not.\sidecite{goodwin_colloids_2009}
Usually, surfactants consist of a hydrocarbon tail, which is hydrophobic, and some hydrophilic head group, which can be ionic or non-ionic.
When surfactants are present in water, the two components will interact differently with the solvent.
A hydration sphere of water molecules will form around the hydrophilic head group, effectively allowing the head group to take part in the water's hydrogen-bonding network.
Whereas, the lyophilic tail has a structure-breaking effect on the hydrogen bonding network, termed the ``hydrophobic effect''.
The free energy deficit of this structure-breaking can be reduced through the aggregation of these hydrophobic groups, as the van der Waals attraction between tail groups is larger than that present between tail groups and water molecules.
There is a decrease in entropy from the tail organisation, however, this is offset by the entropic increase from the water structure breakup.
Finally, by considering the effect of the, often charged, head groups being close together, it is thought that the majority of the charge can be screened by the presence of a counter-ion, or water molecules, bound to the head group.\sidecite{goodwin_colloids_2009}
This means that at low concentrations, where it is statistically unlikely for an agglomerate to form, there will be a higher concentration of surfactants at the air-water interface, and as the concentration is increased, assuming the system is above the Krafft temperature,\footnote{The lowest temperature at which agglomerates will form.} organised structures will begin to appear in solution.

The structures that can be formed from surfactant solutions are diverse; featuring micellar, hexagonal, cubic, and lamellar mesophases.
These mesophases have a significant impact on the macroscopic properties of the system, for example, the liquid crystalline hexagonal phase can present interesting viscoelastic behaviour.\sidecite{jurasin_lamellar_2013,cordobes_linear_1997}
The mesophase that is formed is dependent on the shape of the underlying surfactants.
Israelachvili described this dependency in terms of the dimensionless surfactant packing parameter, $p$.\sidecite{israelachvili_intermolecular_2011}
%
\begin{equation}
p = \frac{V_c}{a_0l_0},
\end{equation}
%
where, $V_c$ is the volume of the hydrophobic tail, $l_0$ is the length of the tail, and $a_0$ is the optimum head group area.
This parameter can be used to estimate the geometry of the resulting self-assembled structure, detailed in Figure~\ref{fig:pack}.
It is important to note that the optimum head group area accounts for the hydration sphere of the head group.
A short tail surfactant\footnote{Such as \emph{n}-decyltrimethylammonium bromide.} will have a very small packing parameter resulting in small spherical micelles.
Whereas, the twin-tailed phospholipids, such as 1,2-dipalmitoyl-\emph{sn}-glycero-3-phosphocholine,\footnote{Known as DPPC.} will have a much larger packing parameter due to the larger tail volume and length, therefore this surfactant will form a lamellar bilayer in solution.
%
\begin{figure}[t]
    \centering
    \includegraphics[width=\textwidth]{introduction/surf_pack}
    \caption{A graphical representation of the packing parameters and information of the resulting self-assembled structure.}
    \label{fig:pack}
\end{figure}
%

This work will focus on the investigation of surfactant monolayers and micellar systems.
These represent interesting model systems of significant relevance both technologically\sidecite{anton_reverse_2010,zagnoni_miniaturised_2012} and biologically.\sidecite{kataoka_block_2012,mohwald_phospholipid_1990,kewalramani_effects_2010}
Both of these systems are regularly investigated using X-ray and neutron elastic scattering techniques, with analysis performed in a model-dependent fashion.\sidecite{pambou_structural_2015,hayward_liquid_2015,rodriguez-loureiro_neutron_2017,hazell_langmuir_2016}
