Soft matter is an umbrella term for many different types of material.
These include micelles; sub-micron sized, dynamic agglomerates of amphiphilic molecules such as surfactants or block co-polymers, colloidal solutions; where the interaction between the the colloids may be controlled through chemical modification, or proteins; where the polar nature of different amino acids leads to the protein folding into a highly organised, and biologically relevant, shape.
Examples of soft matter systems are shown in Figure~\ref{fig:soft}.
These species, initially, appear rather disparate, however, there are a few important commonalities among soft matter systems \cite{jones_soft_2002}:
\begin{itemize}
  \item the lengths scales are intermediate between atomistic and macroscopic; typically in \SIrange{1e-8}{1e-5}{\meter},
  \item for soft matter systems the energy of a structural distortion is similar to thermal energy, so the material in solution is in constant flux,
  \item this thermal motion can lead to the formation of complex, hierarchical structures due to the balance between enthalpy and entropy, this process is refered to as self-assembly.
\end{itemize}
%
\begin{figure}
    \centering
    \includegraphics[width=0.80\textwidth]{introduction/soft_matter_examples}
    \caption{Three examples of soft matter species; (a) a 43 \ce{C_{10}TAB} surfactant micelle \cite{hargreaves_atomistic_2011}, (b) the tunable interactions of colloids \cite{kraft_patchy_2011}, and (c) the crystal structure of T4-lysozyme \cite{rose_crystal_1988}.}
    \label{fig:soft}
\end{figure}
%

\section{Soft matter self-assembly}

Soft matter self-assembly is the ability for soft matter systems to form organised structures in solution.
These are of particular interest industrially, where surfactant and polymer self-assembly plays an import role in food, commodity, and speciality chemicals \cite{schramm_surfactants_2003}.
Self-assembly processes are important from a biological perspective as it is phospholipids, a family of surface-active biomolecules, which make up the bilayers that protect cells \cite{simons_lipid_2000}.
The structures that result from the self-assembly of soft matter species have fluid-like properties.
This is due to the fact that the subunits are held together by weak forces such as the van der Waals, hydrophobic, hydrogen-bonding, and screen electrostatic interactions \cite{israelachvili_intermolecular_2011}.
This means that the structure of a self-assembled species is susceptible to changes in the local chemical environment, such as pH or salt concentration \cite{schmaljohann_thermo-_2006,sammalkorpi_ionic_2009}.

The focus of this work is on the self-assembly of surfactant molecules.
Surfactant is a general term for any molecule which is \emph{surface-active}, that is it will interact at an interface \cite{rosen_surfactants_2012}.
Surfactants are generally made up of two components; one part is highly soluble in one of the interfacial phases, while the other is not \cite{goodwin_colloids_2009}.
Usually, surfactants consist of a hydrocarbon tail, which is hdrophobic, and some hydrophilic head groups, which can be ionic or non-ionic.
When surfactants are present in water, the two components will interact different with the solvent.
A hydration sphere of water molecules will form around the hydrophilic head group, effectively allowing the head group to take part in the hydrogen-bonding network of the water.
Whereas, the lyophilic tail has a structure-breaking effect on the hydrogen bonding network, termed the ``hydrophobic effect''.
The free energy deficit of this structure-breaking can be reduced through the aggregation of these hydrophobic groups, as the van der Waals attraction between tail groups is larger than between tail groups and water molecules.
There is a decrease in entropy from the tail organisation, however, this is offset by the entropic increase from the water structure breakup.
Finally, by considering the effect of the, often charged, head groups being close together, it is thought that the majority of the charge can be screened by the presence of a counter-ion, or water molecules, bound to the head group \cite{goodwin_colloids_2009}.
This means that at low concentrations, where it is statistically unlikely for an agglomerate to form, the majority of surfactants will sit at the air-water interface, as the concentration is increased, assuming the system is above the Krafft temperature (the lowest temperature at which agglomerated will form), organised structures wil begin to appear.

%\subsubsection{Thermodynamics of self-assembly}

%The work of Tanford developed the theoretical understanding of the thermodynamics of a self-assembly process \cite{tanford_hydrophobic_1980}, this has then been applied to a wide variety of soft matter systems, such as micelles, bilayers, and microemulsions.
%For any system that is self-assembling in solution, it is necessary that all identical molecules have the same chemical potential, $\mu$.
%This can be expressed as,
%
%\begin{equation}
%\mu = \mu_N = \mu_N^{\circ} + \frac{k_BT}{N} \ln \Bigg(\frac{X_N}{N}\Bigg) = \text{constant},\;\;\;N = 1,\;2,\;3,\ldots,
%\end{equation}
%
%where, $\mu_N$ is the mean chemical potential of a molecule in an aggregate of aggregation number $N$, $\mu_N^{\circ}$ is the mean interaction free energy er molecule in aggregates of aggregation number $N$, and $X_N$ is the concentration of molecules in aggregates of aggregation number $N$.
%From Figure~\ref{fig:thermo} it is possible to describe the rates of association and dissociation of monomers as follows,
%
%\begin{equation}
%\begin{aligned}
%\text{rate of association} & k_1(X_1)^N \\
%\text{rate of dissociation} & k_N\Bigg(\frac{X_N}{N}\Bigg).
%\end{aligned}
%\end{equation}
%
%These rates allow for the definition of an equilibrium constant, $K$ for the self-assembly process,
%
%\begin{equation}
%K = \frac{k_1}{k_N} = \exp{\Bigg[\frac{N(\mu_N^\circ - \mu_1^\circ)}{k_BT}\Bigg]}
%\end{equation}
%
%This allows for Equation~\ref{equ:thermo} to be rewritten in a more useful form,
%
%\begin{equation}
%X_N = N \Bigg[X_1\exp\Bigg(\frac{\mu_1^\circ - \mu_N^\circ}{k_BT}\Bigg)\Bigg]^N,
%\label{equ:use}
%\end{equation}
%
%between Equation~\ref{equ:use} and ther conservation relation for the total solute concentration, $C$, the system is completely defined, where,
%
%\begin{equation}
%C = \sum_{N=1}^{\infty}X_N.
%\end{equation}
%
%It should however be noted that this assumed that there is no inter-agglomerate interactions, in scattering terms this means no structure factor.
%
%\begin{figure}
%    \centering
%    \includegraphics[width=0.80\textwidth]{introduction/surf_thermo}
%    \caption{Definitions of parameters for the thermodynamic description of self-assembly, from Ref \cite{israelachvili_intermolecular_2011}.}
%    \label{fig:thermo}
%\end{figure}
%

The structures that are formed from self-assembled surfactant systems are diverse; featuring micellar, hexagonal, cubic, and lamallar mesophases.
These mesphases have a significant impact on the macroscopic properties of the system, for example the liquid crystalline hexagonal phase can have interesting viscoelastic behaviour \cite{jurasin_lamellar_2013,cordobes_linear_1997}.
The mesophase that is formed is dependent on the shape of the underlying surfactants, Israelachvili described this dependency in terms of the dimensionless surfactant packing parameter \cite{israelachvili_intermolecular_2011},
%
\begin{equation}
p = \frac{V_c}{a_0l_0},
\end{equation}
%
where, $V_c$ is the volume of the hydrophobic tail, $l_0$ is the length of the tail, and $a_0$ is the optimum head group area.
This parameter can be used to estimate the geometry of the resulting self-assembled structure, detailed in Figure~\ref{fig:pack}.
It is important to note that the optimum head group area accounts for the hydration sphere of the head group.
A short tail surfactant, such as \emph{n}-decyltrimethylammounium bromide, will have a very small packing parameter resulting in small spherical micelles.
Whereas, the twin-tailed phospholipids, such as 1,2-dipalmitoyl-\emph{sn}-glycero-3-phosphocholine (DPPC), will have a much larger packing parameter due to the larger tail volume and length, therefore this surfactant will for a lamellar bilayer in solution.
%
\begin{figure}
    \centering
    \includegraphics[width=0.80\textwidth]{introduction/surf_pack}
    \caption{A graphical representation of the packing parameters and information of the resulting self-assembled structure, adapted from \cite{israelachvili_intermolecular_2011}.}
    \label{fig:pack}
\end{figure}
%

This work will focus on the investigation of surfactant monolayers and micellar systems.
These represent interesting model systems of significant interest both technologically \cite{anton_reverse_2010,zagnoni_miniaturised_2012} and biologically \cite{kataoka_block_2012,mohwald_phospholipid_1990,kewalramani_effects_2010}.
Both of these systems are regularly investigated using X-ray and neutron elastic scattering techniques, with analysis performed in a model-dependent fashion \cite{pambou_structural_2015,hayward_liquid_2015,rodriguez-loureiro_neutron_2017,hazell_langmuir_2016}.
