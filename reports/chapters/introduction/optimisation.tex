\section{Optimisation methodologies}

The availability of high performance computing has increased significantly in recent year, in particular due to cloud-based infrastructures.
Furthermore, highly parallelisable optimisation algorithms are now available such as the particle swarm optimisation \cite{kennedy_particle_1995,shi_modified_1998} and the differtial evolution \cite{storn_differential_1997}.
As mentioned above, previous work has shown that the simulation of a surfactant micelle and comparison with experimental data requires significant computational expense \cite{hargreaves_atomistic_2011,ivanovic_temperature-dependent_2018}.
In the interest of reducing this expense, and improving the applicibility of high performance computing to the simulation driven analysis of small angle scattering I have investigated the use of particle swarm optimisation to produce a realistic near-atomistic micelle structure based on experimental data alone.
This has made use of a coarse-grained description of a surfactant molecule on two levels; one for the particle swarm optimisation and another for the potential energy minimisation (Chapter~\ref{smallangle}).
