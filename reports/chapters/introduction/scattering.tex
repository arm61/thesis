\section{Analysis of soft matter scattering}
The use of neutron and X-ray scattering experiments for the study of soft matter is well developed, with early research into the structure of phospholipid monolayers by reflectometry methods being conducted in the late 1970s by Albrecht \emph{et al.}\autocite{albrecht_polymorphism_1978}
While, the work of Kratky and Porod,\autocite{kratky_diffuse_1949} who used small angle X-ray scattering for the study of colloidal systems was published in 1949.
Since these early works, instrumentation developments have enabled more challenging experiments to be conducted, such as time-resolved studies \autocite{jensen_monitoring_2014} and the study of floating phospholipid bilayers.\autocite{rondelli_reflectivity_2012}

However, the analysis of soft matter scattering has changed little since these early works, still typically involving the use of very coarse models.
These include the shape-based modelling common in small angle scattering \autocite[][see Section~\ref{sec:sasanal}]{hassan_small_2003} and reflectometry analysis.\autocite[][see Section~\ref{sec:sasanal}]{campbell_structure_2018,lu_analysis_1996}
More sophisticated model refinements have been developed, such as the use of Monte-Carlo sampling,\autocite{pedersen_monte_2002} differential evolution optimisation,\autocite{wormington_characterization_1999} and Bayesian inference.\autocite{nelson_refnx_2019}
However, there has been little change in the definition of the models that unpin the analysis processes.
Recently, there have been movements towards the use of atomistic modelling techniques\footnote{Such as molecular dynamics.} to augment, and assist, the analysis of soft matter scattering measurements, in a multi-modal approach.\autocite{scoppola_combining_2018}

Much of the work relating to the use of atomistic simulation for the analysis of small angle scattering measurements has been focused on the study of protein molecules in solution.\footnote{The historical context of this is discussed briefly in Chapter~\ref{smallangle}.}
This has allowed for a more profound understanding aspects of biology suhc as the conformational states available to protein molecules in solution.\autocite{bowerman_determining_2017}
The uptake of atomistic simulation for the analysis of small angle scattering from systems such as micelles has been slower, in part due to the more complex conformation landscape available to these systems under standard conditions.
However, the work of Hargreaves \emph{et al.} paired atomistic simulation with total scattering measurements\footnote{In the form of Emperical Potential Structure Refinement.} to resolve the structure of an simple short-tail surfactant micelle.\autocite{hargreaves_atomistic_2011}
Further, the work of Ivanovi\'{c} \emph{et al.} used scattering experiments to refine the output of molecular dynamics simulations of micelles of a pre-defined size.\autocite{ivanovic_temperature-dependent_2018}
Both of these examples required significant computational resource; in the former case, the computational time taken was quoted as 200 days, while the later required the running of multiple simulations at different micelle sizes in order to determine the appropriate simulation.

The use of atomistic simulation for the analysis reflectometry measurements of soft matter systems began with the work of Miller \emph{et al.} and Anderson and Wilson,\autocite{miller_monte_2003,anderson_molecular_2004} where atomistic simulations\footnote{Monte Carlo and molecular dynamics respectively.} were used to study polymer self-assembly at the oil-water interface.
These simulation trajectories was then compared with experimental neutron reflectometry measurements.
Dabkowska \emph{et al.} also used atomistic simulation and neutron reflectometry measurements to study the structure of a surfactant monolayer at the air-water interface, providing the first example of a direct comparison between experimental reflectometry data and that determined from simulation.\autocite{dabkowska_modulation_2014}
To date, there is only one work that has used coarse-grained molecular dynamics simulation to aid in the analysis of neutron reflectometry, this is the work of Koutsioubas.\autocite{koutsioubas_combined_2016}
This work made use of the MARTINI coarse-grained forcefield to simulate a phospholipid bilayer, and was compared with experimental neutron reflectometry measurements.
