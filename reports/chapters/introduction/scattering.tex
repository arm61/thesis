\section{Soft matter scattering}

The use of scattering experiments for the study of soft matter is well developed, with early research into the structure of phospholipid monolayers by reflectometry being conducted in the layer 1970s by Albrecht \emph{et al.} \cite{albrecht_polymorphism_1978}.
While the work of Kratky and Porod \cite{kratky_diffuse_1949}, who used small angle X-ray scattering for the study of colloidal systems was conducted in 1949.
The analysis of these scattering methods, since early applications, have typically involved the use of very coarse models.
These include the shape-based modelling common in small angle scattering (discussed in Section~\ref{sec:sasanal}) \cite{hassan_small_2003} and the layer models (see Section~\ref{sec:refltheory}) \cite{campbell_structure_2018} and functional models \cite{lu_analysis_1996} used in reflectometry analysis.
However, recent developments in atomistic modelling techniques, such as molecular dynamics, have lead to significant interest in the use of modelling as a method for the multi-modal analysis of soft matter scattering measurements \cite{scoppola_combining_2018}.
