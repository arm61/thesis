\section{Coarse-graining of soft matter systems}
The characteristic non-atomistic length scales associated with soft matter systems make them ideal for the application of coarse-graining protocols.
Coarse-graining is where dimensionality of a problem is reduced by the removal of certain degrees of freedom from a set.
The most common method of coarse-graining is the re-parameterisation of an atomistic molecular dynamics potential model in terms of this reduced parameter space.
An example of this is the MARTINI forcefield,\footnote{This specific model is discussed in greater detail in Section~\ref{sec:coarsegraining}.} where the aim is to reparameterise the system without significant loss of chemical information.\autocite{marrink_martini_2007}
A result of coarse-graining is the creation a flatter potential energy landscape, as shown in Figure~\ref{fig:cg}.
However, in this work I have also investigated the effect of applying a chemically-consistent coarse-grained monolayer model for the analysis of reflectometry data,\footnote{This work is the focus of Chapter~\ref{reflectometry1}.}
This system is coarse-grained so as to describe a phospholipid material as consisting of a head group and pair of tail groups.
Additionally, I have assessed the efficacy of different atomistic and coarse-grained potential models for the analysis of neutron reflectometry, building on the work of Dabkowska \emph{et al.} and Koutsioubas.\autocite[][see Chapter~\ref{reflectometry2}]{dabkowska_modulation_2014,koutsioubas_combined_2016}
%
\begin{marginfigure}
    \includegraphics[width=\linewidth]{introduction/cg}
    \caption{Potential energy surfaces for an all-atom vs a coarse-grained potential model, reprinted with permission of the American Chemical Society from \cite{kmiecik_coarse-grained_2016}.}
    \label{fig:cg}
\end{marginfigure}
%
