\section{Coarse-graining of soft matter systems}

The characteristic non-atomistic length scales associated with soft matter systems makes them ideal for the application of coarse-graining protocols.
Coarse-graining is where dimensionality of a problem is reduced by the removal of certain degrees of freedom from a set.
The most common method of coarse-graining is the re-parameterisation of an atomistic molecular dynamics potential model in terms of this reduced parameter space.
An example of this is the MARTINI forcefield, discussed in detail in Section~\ref{sec:coarsegraining}, where the system is parameterised without loss of chemical information.
The result of coarse-graining is to create a flatter potential energy landscape, as shown in Figure~\ref{fig:cg}.
However, in this work I have also investigated the effect of applying a chemically-consistent coarse-grained monolayer model for the analysis of reflectometry data (Chapter~\ref{reflectometry1}), where the system is coarse-grained to describe a phospholipid material as consisting of a head group and pair of tail groups.
In addition to the assessment of the severe coarse-graining protocol, I have assessed the efficacy of different atomistic and coarse-grained potential models for the analysis of neutron reflectometry, building on the work of Dabkowska \emph{et al.} and Koutsioubas \cite{koutsioubas_combined_2016} (Chapter~\ref{reflectometry2}).
%
\begin{figure}
    \centering
    \includegraphics[width=0.40\textwidth]{introduction/cg}
    \caption{Potential energy surfaces for an all-atom vs a coarse-grained potential model, from \cite{kmiecik_coarse-grained_2016}.}
    \label{fig:pack}
\end{figure}
%
