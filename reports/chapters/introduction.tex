% Chapter Template

\chapter{Introduction} % Main chapter title

\label{introduction} % Change X to a consecutive number; for referencing this chapter elsewhere, use \ref{ChapterX}

%----------------------------------------------------------------------------------------
%	SECTION 1
%----------------------------------------------------------------------------------------

The aim of this work is to investigate a series of different coarse-graining methodologies that can be used in the analysis of scattering data from soft matter.
The coarse-graining used in this work includes traditional potential model coarse-graining,\footnote{Such as the use of the MARTINI potential model.} applying a layer-based coarse-grained description of a surfactant monolayer to the analysis of reflectometry, and developing a severely coarse-grained description of a surfactant molecule to allow the easy generation of realistic starting structures\footnote{For use in classical simulation.} from experimental data.
From this work, I hope that those interested in using coarse-grained approaches for the analysis of scattering data will be able to better understand the avenues that are open to them and the possible pitfalls that are present. 

\section{Soft Matter}

Soft matter is an umbrella term for many different types of material.
These include micelles; sub-micron sized, dynamcsi agglomerates of amphiphilic molecules such as surfactants or block co-polymers, colloidal solutions; where the interaction between the the colloids may be controlled through chemical modification, or proteins; where the polar nature of different amino acids leads to the protein folding into a highly organised, and biologically relevant, shape.
Examples of soft matter systems are shown in Figure~\ref{fig:soft}.
These species, initially, appear rather disparate, however, there are a few important commonalities among soft matter systems \cite{jones_soft_2002}:
\begin{itemize}
\item the lengths scales are intermediate between atomistic and macroscopic; typically in \SIrange{1e-8}{1e-5}{\meter},
\item for soft matter systems the energy of a structural distortion is similar to thermal energy, so the material in solution is in constant flux,
\item this thermal motion can lead to the formation of complex, hierarchical structures due to the balance between enthalpy and entropy, this process is refered to as self-assembly.
\end{itemize}
%
\begin{figure}
    \centering
    \includegraphics[width=0.80\textwidth]{introduction/soft_matter_examples}
    \caption{Three examples of soft matter species; (a) a 43 \ce{C_{10}TAB} surfactant micelle \cite{hargreaves_atomistic_2011}, (b) the tunable interactions of colloids \cite{kraft_patchy_2011}, and (c) the crystal structure of T4-lysozyme \cite{rose_crystal_1988}.}
    \label{fig:soft}
\end{figure}
%

\subsection{Self-assembly}

Soft matter self-assembly is the ability for soft matter systems to form organised structures in solution.
These are of particular interest industrially, where surfactant and polymer self-assembly plays an import role in food, commodity, and speciality chemicals \cite{schramm_surfactants_2003}.
Self-assembly processes are important from a biological perspective as it si phospholipids, a family of surface-active biomolecules, which make up the bilayers that protect cells \cite{simons_lipid_2000}.
The structures that result from the self-assembly of soft matter species have fluid-like properties.
This is due to the fact that the subunits are held together by weak forces such as the van der Waals, hydrophobic, hydrogen-bonding, and screen electrostatic interactions \cite{israelachvili_intermolecular_2011}.
This means that the structure of a self-assembled species is susceptible to changes in the local chemical environment, such as pH or salt concentration \cite{schmaljohann_thermo-_2006,sammalkorpi_ionic_2009}.

The focus of this work is on the self-assembly of surfactant molecules.
Surfactant is a general term for any molecule which is \emph{surface-active}, that is it will interact at an interface \cite{rosen_surfactants_2012}.
Surfactants are generally made up of two components; one part is highly soluble in one of the interfacial phases, while the other is not \cite{goodwin_colloids_2009}.
Usually, surfactants consist of a hydrocarbon tail, which is hdrophobic, and some hydrophilic head groups, which can be ionic or non-ionic.
When surfactants are present in water, the two components will interact different with the solvent.
A hydration sphere of water molecules will form around the hydrophilic head group, effectively allowing the head group to take part in the hydrogen-bonding network of the water.
Whereas, the lyophilic tail has a structure-breaking effect on the hydrogen bonding network, termed the ``hydrophobic effect''.
The free energy deficit of this structure-breaking can be reduced through the aggregation of these hydrophobic groups, as the van der Waals attraction between tail groups is larger than between tail groups and water molecules.
There is a decrease in entropy from the tail organisation, however, this is offset by the entropic increase from the water structure breakup.
Finally, by considering the effect of the, often charged, head groups being close together, it is thought that the majority of the charge can be screened by the presence of a counter-ion, or water molecules, bound to the head group \cite{goodwin_colloids_2009}.
This means that at low concentrations, where it is statistically unlikely for an agglomerate to form, the majority of surfactants will sit at the air-water interface, as the concentration is increased, assuming the system is above the Krafft temperature (the lowest temperature at which agglomerated will form), organised structures wil begin to appear.

\subsubsection{Thermodynamics of self-assembly}

The work of Tanford developed the theoretical understanding of the thermodynamics of a self-assembly process \cite{tanford_hydrophobic_1980}, this has then been applied to a wide variety of soft matter systems, such as micelles, bilayers, and microemulsions.
For any system that is self-assembling in solution, it is necessary that all identical molecules have the same chemical potential, $\mu$.
This can be expressed as,
%
\begin{equation}
\mu = \mu_N = \mu_N^{\circ} + \frac{k_BT}{N} \ln \Bigg(\frac{X_N}{N}\Bigg) = \text{constant},\;\;\;N = 1,\;2,\;3,\ldots,
\end{equation}
%
where, $\mu_N$ is the mean chemical potential of a molecule in an aggregate of aggregation number $N$, $\mu_N^{\circ}$ is the mean interaction free energy er molecule in aggregates of aggregation number $N$, and $X_N$ is the concentration of molecules in aggregates of aggregation number $N$.
From Figure~\ref{fig:thermo} it is possible to describe the rates of association and dissociation of monomers as follows,
%
\begin{equation}
\begin{aligned}
\text{rate of association} & k_1(X_1)^N \\
\text{rate of dissociation} & k_N\Bigg(\frac{X_N}{N}\Bigg).
\end{aligned}
\end{equation}
%
These rates allow for the definition of an equilibrium constant, $K$ for the self-assembly process,
%
\begin{equation}
K = \frac{k_1}{k_N} = \exp{\Bigg[\frac{N(\mu_N^\circ - \mu_1^\circ)}{k_BT}\Bigg]}
\end{equation}
%
This allows for Equation~\ref{equ:thermo} to be rewritten in a more useful form,
%
\begin{equation}
X_N = N \Bigg[X_1\exp\Bigg(\frac{\mu_1^\circ - \mu_N^\circ}{k_BT}\Bigg)\Bigg]^N,
\label{equ:use}
\end{equation}
%
between Equation~\ref{equ:use} and ther conservation relation for the total solute concentration, $C$, the system is completely defined, where,
%
\begin{equation}
C = \sum_{N=1}^{\infty}X_N.
\end{equation}
%
It should however be noted that this assumed that there is no inter-agglomerate interactions, in scattering terms this means no structure factor.
%
\begin{figure}
    \centering
    \includegraphics[width=0.80\textwidth]{introduction/surf_thermo}
    \caption{Definitions of parameters for the thermodynamic description of self-assembly, from Ref \cite{israelachvili_intermolecular_2011}.}
    \label{fig:thermo}
\end{figure}
%

\subsubsection{Packing parameters}

The structures that are formed from self-assembled surfactant systems are diverse; featuring micellar, hexagonal, cubic, and lamallar mesophases.
These mesphases have a significant impact on the macroscopic properties of the system, for example the liquid crystalline hexagonal phase can have interesting viscoelastic behaviour \cite{jurasin_lamellar_2013,cordobes_linear_1997}.
The mesophase that is formed is dependent on the shape of the underlying surfactants, Israelachvili described this dependency in terms of the dimensionless surfactant packing parameter \cite{israelachvili_intermolecular_2011},
%
\begin{equation}
p = \frac{V_c}{a_0l_0},
\end{equation}
%
where, $V_c$ is the volume of the hydrophobic tail, $l_0$ is the length of the tail, and $a_0$ is the optimum head group area.
This parameter can be used to estimate the geometry of the resulting self-assembled structure, detailed in Figure~\ref{fig:pack}.
It is important to note that the optimum head group area accounts for the hydration sphere of the head group.
A short tail surfactant, such as \emph{n}-decyltrimethylammounium bromide, will have a very small packing parameter resulting in small spherical micelles.
Whereas, the twin-tailed phospholipids, such as 1,2-dipalmitoyl-\emph{sn}-glycero-3-phosphocholine (DPPC), will have a much larger packing parameter due to the larger tail volume and length, therefore this surfactant will for a lamellar bilayer in solution. 
%
\begin{figure}
    \centering
    \includegraphics[width=0.80\textwidth]{introduction/surf_pack}
    \caption{A graphical representation of the packing parameters and information of the resulting self-assembled structure, adapted from \cite{israelachvili_intermolecular_2011}.}
    \label{fig:pack}
\end{figure}
%

\section{Analysis of soft matter scattering}
The use of neutron and X-ray scattering experiments for the study of soft matter is well developed, with early research into the structure of phospholipid monolayers by reflectometry methods being conducted in the late 1970s by Albrecht \emph{et al.}\sidecite<-1.5\baselineskip>{albrecht_polymorphism_1978}
While, the work of Kratky and Porod,\sidecite<-1\baselineskip>{kratky_diffuse_1949} who used small angle X-ray scattering\footnote{Generally abbreviated to SAS; with SAXS indicating the use of X-rays and SANS neutrons} for the study of colloidal systems was published in 1949.
Since these early works, instrumentation developments have enabled more challenging experiments to be conducted, such as time-resolved studies \autocite{jensen_monitoring_2014} and the study of floating phospholipid bilayers.\autocite{rondelli_reflectivity_2012}

However, the analysis of soft matter scattering has changed little since these early works, still typically involving the use of very coarse models.
These include the shape-based modelling common in SAS \autocite[][see Section~\ref{sec:sasanal}]{hassan_small_2003} and reflectometry analysis.\autocite[][see Section~\ref{sec:sasanal}]{campbell_structure_2018,lu_analysis_1996}
More sophisticated model refinements have been developed, such as the use of Monte-Carlo sampling,\autocite{pedersen_monte_2002} differential evolution optimisation,\autocite[abbreviated to DE]{wormington_characterization_1999} and Bayesian inference.\autocite{nelson_refnx_2019}
However, there has been little change in the definition of the models that unpin the analysis processes.
Recently, there have been movements towards the use of atomistic modelling techniques\footnote{Such as molecular dynamics (MD).} to augment, and assist, the analysis of soft matter scattering measurements, in a multi-modal approach.\autocite{scoppola_combining_2018}

Much of the work relating to the use of atomistic simulation for the analysis of SAS measurements has been focused on the study of protein molecules in solution.\footnote{The historical context of this is discussed briefly in Chapter~\ref{smallangle}.}
This has allowed for more profound understanding aspects of biology such as the conformational states available to protein molecules in solution.\autocite{bowerman_determining_2017}
The uptake of atomistic simulation for the analysis of SAS from systems such as micelles has been slower, in part due to the more complex conformation landscape available to these systems under standard conditions.
However, the work of Hargreaves \emph{et al.} paired atomistic simulation with total scattering measurements\footnote{In the form of Empirical Potential Structure Refinement.} to resolve the structure of a simple short-tail surfactant micelle.\autocite{hargreaves_atomistic_2011}
Further, the work of Ivanovi\'{c} \emph{et al.} used scattering experiments to refine the output of MD simulations of micelles of a pre-defined size.\autocite{ivanovic_temperature-dependent_2018}
Both of these examples required significant computational resource; in the former case, the computational time taken was quoted as 200 days, while the later required the running of multiple simulations at different micelle sizes in order to determine the appropriate simulation.

The use of atomistic simulation for the analysis reflectometry measurements of soft matter systems began with the work of Miller \emph{et al.} and Anderson and Wilson,\autocite{miller_monte_2003,anderson_molecular_2004} where atomistic simulations\footnote{Monte Carlo and MD respectively.} were used to study polymer self-assembly at the oil-water interface.
These simulation trajectories were then compared with experimental neutron reflectometry\footnote{Abbreviated to NR.} measurements.
Dabkowska \emph{et al.} also used atomistic simulation and NR measurements to study the structure of a surfactant monolayer at the air-water interface, providing the first example of a direct comparison between experimental reflectometry data and that determined from simulation.\autocite{dabkowska_modulation_2014}
To date, there is only one work that has used coarse-grained MD simulation to aid in the analysis of NR, this is the work of Koutsioubas.\autocite{koutsioubas_combined_2016}
This work made use of the MARTINI coarse-grained potential model to simulate a phospholipid bilayer and was compared with experimental NR measurements.

\section{Coarse-graining of soft matter systems}
The characteristic non-atomistic length scales associated with soft matter systems make them ideal for the application of coarse-graining protocols.
Coarse-graining is where dimensionality of a problem is reduced by the removal of certain degrees of freedom from a set.
The most common method of coarse-graining is the re-parameterisation of an atomistic molecular dynamics potential model in terms of this reduced parameter space.
An example of this is the MARTINI forcefield,\footnote{This specific model is discussed in greater detail in Section~\ref{sec:coarsegraining}.} where the aim is to reparameterise the system without significant loss of chemical information.\autocite{marrink_martini_2007}
A result of coarse-graining is the creation a flatter potential energy landscape, as shown in Figure~\ref{fig:cg}.
The availability of coarse-grained potential models and tools for coarse-graining as allowed for very large simulations to be performed, such as the simulation of large polymeric species,\autocite{carbone_transferability_2008} phosphoplipid nanodiscs,\autocite{xue_molecular_2018} and realistic biomembranes.\autocite{marrink_computational_2019}
Furthermore, these coarse-grained potential model have shown the ability to reproduce experimental measurements.\autocite{negro_experimental_2014,nawaz_interactions_2012}

However, in this work I have used the term ``coarse-graining'' broadly to include the applications of a chemically-consistent coarse-grained monolayer model for the analysis of reflectometry data,\footnote{This work is the focus of Chapter~\ref{reflectometry1}.}
In addition to the assessment of different atomistic and coarse-grained potential models for the analysis of neutron reflectometry, building on the work of Dabkowska \emph{et al.} and Koutsioubas.\autocite[][see Chapter~\ref{reflectometry2}]{dabkowska_modulation_2014,koutsioubas_combined_2016}
%
\begin{marginfigure}
    \includegraphics[width=\linewidth]{introduction/cg}
    \caption{Potential energy surfaces for an all-atom vs a coarse-grained potential model, reprinted with permission of the American Chemical Society from \cite{kmiecik_coarse-grained_2016}.}
    \label{fig:cg}
\end{marginfigure}
%

\section{Optimisation methodologies}

The availability of high performance computing has increased significantly in recent year, in particular due to cloud-based infrastructures.
Furthermore, highly parallelisable optimisation algorithms are now available such as the particle swarm optimisation \cite{kennedy_particle_1995,shi_modified_1998} and the differential evolution \cite{storn_differential_1997}.
As mentioned above, previous work has shown that the simulation of a surfactant micelle and comparison with experimental data requires significant computational expense \cite{hargreaves_atomistic_2011,ivanovic_temperature-dependent_2018}.
In the interest of reducing this expense, and improving the applicibility of high performance computing to the simulation driven analysis of small angle scattering I have investigated the use of particle swarm optimisation to produce a realistic near-atomistic micelle structure based on experimental data alone.
This has made use of a coarse-grained description of a surfactant molecule on two levels; one for the particle swarm optimisation and another for the potential energy minimisation (Chapter~\ref{smallangle}).

\section{Educational materials}
While the development of analytical methods and infrastructure are important for the development and uptake of the simulation-driven analysis methods applied in this work, the development of informatative resources is also necessary.
Current experimental users of small angle scattering are typically not familiar with detailed aspects of classical simulation, however, they are often interested in applying it to assist with their analyses.
Therefore, alongside traditional research applications, I have been developing open educational resources\footnote{Abbreviated to OERs.} designed to introduce classical simulation techniques, and to allow users of scattering techniques to become familiar with these.
The ambition being that as the availablility of simulation-driven analysis for small angle scattering grows, so will the user base that is familiar with the underlying methods.\footnote{See Chapter~\ref{teaching}.}

