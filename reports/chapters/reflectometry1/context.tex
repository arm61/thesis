\section*{Context}
This project offers a severe method of coarse-graining for the analysis of neutron and X-ray reflectometry data.
The system in coarse-grained to exist as a head group and a tail group of a phospholipid species.
There is a chemical constraint present in the model, specifically that the number of head groups must be equal to the number of pairs of tail groups.
However, there is no potential model considered beyond this ``bonded'' interaction.
Additionally, this modelling approach is applied again in Chapter~\ref{reflectometry2}, as an example of the cutting edge of traditional modelling, against which the classical potential model-driven methods will be compared.
The specific application of this modelling approach grew from a collaboration with experimental colleagues working on self-assembly in deep eutectic solvents.
Therefore, this chemical system will be briefly introduced in Section~\ref{sec:ref1intro}.
However, the main focus of this chapter will be the modelling methodology, therefore, this will remain the focus of this chapter.

\subsection*{Publications}
Parts of this work have been published in , a copy of the publication may be found in Appendix~\ref{papers}.
