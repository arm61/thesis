\section{Introduction}
\label{sec:ref1intro}
\subsection{Deep eutectic solvents}
DES are a class of green, sustainable liquids that may be obtained from the combination of ionic species with compounds capable of acting as hydrogen bond donors, such as sugar, alcohols, amines, and carboxylic acids.\autocite{smith_deep_2014,dai_natural_2013}
The resulting extensive hydrogen bonding network is capable of stabilising both species, such that the eutectic mixture will remain liquid at room temperature.\autocite{hammond_liquid_2016,hammond_resilience_2017,araujo_inelastic_2017}
Using different precursor materials can allow for the ability to tune the resulting solvent's physicochemical properties, such as polarity,\autocite{pandey_how_2014} viscosity and surface tension,\autocite{smith_deep_2014} network charge,\autocite{zahn_charge_2016} and hydrophobicity.\autocite{ribeiro_menthol-based_2015,van_osch_hydrophobic_2015}
Recently DES have also been shown to exhibit a ``solvophobic'' effect through the promotion of surfactant micelle formation,\autocite{sanchez-fernandez_micellization_2016,arnold_surfactant_2015,hsieh_micelle_2018,banjare_self-assembly_2018} phospholipid bilayer formation,\autocite{bryant_spontaneous_2016,bryant_effect_2017,gutierrez_freeze-drying_2009} and the ability to stabilise non-ionic polymer\autocite{sapir_properties_2016} and protein conformations.\autocite{sanchez-fernandez_protein_2017}

Phospholipid monolayers at the air/water interface have been widely studied as simplistic models for biological membranes.
As such, they have been used to gain insight into many biological processes that are technologically and medically relevant.
For example, investigations at the air/salt-water interface have identified the importance that interactions between charged phospholipid head groups and ions present in solution have on the structure, monolayer packing and stability.\autocite{mohwald_phospholipid_1990,kewalramani_effects_2010}
However, the native environment for phospholipids in-vivo is far from simple aqueous solutions.
In fact, it has been suggested\autocite{dai_natural_2013,hammond_resilience_2017} that DES might form within the crowded cellular environment and could assist in solubilizing biological species in an intermediate environment between that of the hydrophobic phospholipid tail groups and the highly polar water-rich regions, thereby assisting survival under extreme conditions such as freezing temperatures and drought where the water content of the cells is restricted.

This chapter presents the first observation of phospholipid monolayer at an air-DES interface.
Furthermore, this is one of a few examples of a phospholipid monolayer at the interface between air and a non-aqueous solvent, with only formamide noted previously.\autocite{graner_phospholipidic_1995}
Langmuir monolayers of non-phospholipidic surfactant molecules have also been noted at air-formamide and air-mercury interfaces.\autocite{weinbach_self-assembled_1993,magnussen_self-assembly_1996,kraack_structure_2002}
In these previous works, the authors noted that the non-aqueous surface had an effect on the overall structure of the monolayer, but little was said about the underlying mechanism.

\subsection{Optimisation and sampling in reflectometry analysis}
The analysis of reflectometry data usually involves the use of some model-dependent methodology.
Therefore it is necessary to optimise the difference between our model, and the experimental dataset.
Analytical methods, such as the gradient descent method are not usually suitable for application to the optimisation of a reflectometry model, as these are only capable of optimisation to local minima, which would require accurate prior knowledge of the model structure.\autocite{lovell_analysis_1999}
Despite the analytical nature of the Maximum entropy (MaxEnt) optimisation method, this showed some success in the optimisation of reflectometry models, due in part to the ability of this optimisation to produce of a large number of solutions.\autocite{geoghegan_experimental_1996,bucknall_neutron_1997}
However, this approach is computationally intensive and therefore it is unusual to apply it instead of other more efficient methods.
Some aspects of the MaxEnt methods were replicated in the work of Sivia \emph{et al.}, which employed Bayesian probability theory to rationalise the model selection.\autocite{geoghegan_experimental_1996,sivia_introduction_1993,sivia_bayesian_1998,sivia_analysis_1991}

The use of analytic methods became less favoured as more frequently stochastic methods were used, these offer a more pragmatic solution to the local minima problem.
Stochastic methods are those that utilise inherently random behaviour to determine a global minimum.
The groove tracking method of Zhou and Chen,\autocite{zhou_model-independent_1993,zhou_theoretical_1995} was one of the first examples of a stochastic optimisation process applied to the analysis of reflectometry data.
This randomly varied the SLD of the layers in the model using a Monte Carlo approach.
A similar approach used a simulated annealing approach, with a ``temperature'' factor that decreased as the number of iterations increased,\autocite{kunz_model-free_1993} however, this approach is still subject to the local minima problem as the probability of move acceptance decreased over time.

Both these Monte Carlo based approaches and the analytic methods previously discussed make the same perturbations to the fitted parameters during processing.
However, this is the cause of the propensity to converge to a local minimum.
This led to the application of genetic algorithm-derived methods for the optimisation of reflectometry models, beginning with the works of de Haan and Drijkoningen\sidecite<-6\baselineskip>{de_haan_genetic_1994} and Dane \emph{et al.}\sidecite<-4\baselineskip>{dane_application_1998}
These methods are designed to stochastically sample an entire search-space, and therefore are more able to overcome the local minima issues and determine the vicinity of a global minimum.
Following this initial application, genetic algorithms were used frequently in the optimisation of reflectometry models\sidecite<-7\baselineskip>{ulyanenkov_genetic_2000,ulyanenkov_extended_2005,politsch_unbiased_2002,wormington_characterization_1999}
In particular, the work of Wormington \emph{et al.}\sidecite<-2\baselineskip>{wormington_characterization_1999} showed the applicability of the DE method towards reflectometry model optimisation, which resulted in the inclusion of such methods in many common reflectometry analysis software packages.\sidecite<-3\baselineskip>{bjorck_fitting_2011,bjorck_genx_2007,nelson_co-refinement_2006,nelson_refnx_2019,ott_simulreflec_nodate,kienzle_ncnr_nodate}

The use of MCMC methods to probe the probability distribution functions\sidenote[][-3\baselineskip]{Abbreviated to PDFs.} of the fitting parameters of a reflectometry model have also grown in popularity.\sidecite<-3\baselineskip>{gil_limitations_2012,hoogerheide_structure_2018,owejan_solid_2012,heinrich_myristoylation_2014}
This is due to the inclusion of MCMC methods in common analysis software packages such as Refl1D.\autocite{kienzle_ncnr_nodate}
These methods enable the user to better understand the inverse uncertainties of the model.
Additionally, they enable the quantification of the correlation between parameters, important in ensuring that the model applied is suitably constrained such as to reduce the cross-correlation present between parameters.\autocite{nelson_co-refinement_2006}

This work applies a DE algorithm\autocite{storn_differential_1997,jones_scipy_nodate} to the optimisation of the reflectometry model.
The search-space, available within the experimental uncertainty of the data is then sampled using MCMC, as implemented in \texttt{emcee},\autocite{foreman-mackey_emcee_2013} to understand the parameter probability distributions and quantify the inter-parameter correlations.

\subsection{Chemically-consistent modelling}
The use of chemically-consistent modelling is common in the fitting of XRR and NR measurements from phospholipid monolayers to obtain structural insights.\footnote{The history of this modelling is introduced well in \cite{campbell_structure_2018}.}
While it is possible to model the NR from a phospholipid monolayer with a single layer model,\autocite{wojciechowski_interaction_2016,wojciechowski_complexation_2016} the use of at least two layers; representing the head and tail groups is more commonplace.\autocite{foglia_interaction_2014,bello_influence_2016}
Even when two layers are utilised, it is often the case that the volumes of the head and tail groups, $V_i$, are used as constraints in the modelling process, as the SLD of a layer may be determined as follows,
%
\begin{equation}
\text{SLD}_i = \frac{b_i}{V_i}(1-\phi_i)+\text{SLD}_s(\phi_s)
\label{equ:sld}
\end{equation}
%
where, $b_i$ is the scattering length of the head or tail, $\phi_i$ is the volume fraction of solvation by the solvent, $\text{SLD}_s$ is the solvent scattering length density, and $i$ indicates either the head or tail layer.
However, as noted by Campbell \emph{et al.},\autocite{campbell_structure_2018} this method often fails to account for the compaction of the carbon chains under elevated SPs,\autocite{mcconlogue_close_1997,small_lateral_1984} which may lead to a volume reduction of up to \SI{\sim 15}{\percent}.
Furthermore,\footnote{As is discussed in Section~\ref{refl1:anal}.} the use of a constrained head group volume may also influence the result of the modelling process in situations where the volume is poorly defined.

Equation~\ref{equ:sld} enables the use of chemical-inference in the modelling approach for reflectometry data.
This allows for the co-refinement of NR data where different isotopic-contrasts of the phospholipid species or solvent have been used.
This is possible based on the expectation that the effect of contrast variation on the structure and chemistry of the monolayer will be negligible, and therefore the same values of all parameters in the fitting, except $b_i$ and $\text{SLD}_s$ may be constrained between the different measurements.\autocite{hollinshead_effects_2009}
In this work, similar logic was applied, with the assumption that the volume of the head and tail groups is constant across different SPs, while the phospholipid phase is the same.
This means that, for this system, all of the fitted parameters may be constrained aside from the tail thickness, head solvation, and interfacial roughness, across the different SP measurements.
This is the first time that such a methodology has been applied to the analysis of XRR and NR, additionally, it is believed that this ability to co-refine XRR measurements enables a greater understanding of the structure than that possible from a single measurement.
