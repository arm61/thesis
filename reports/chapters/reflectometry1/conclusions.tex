\section{Conclusions}
Stable PC and PG lipid monolayers were observed and characterised on an ionic solvent surface.
Until the emergence of ionic liquids and DES, only a limited number of molecular solvents exhibited the ability to promote self-assembly and only water and formamide among those had previously demonstrated the formation of phospholipid monolayers at the air-liquid interface.\autocite{mohwald_phospholipid_1990,graner_phospholipidic_1995}

For the first time, a physically and chemically-consistent reflectometry modelling approach was used to co-refine XRR measurements at different surface pressures.
This enabled modelling without the need to constrain the head and tail group volumes, enabling these parameters to vary freely to account for any variation occurring due to the elevated surface pressures used or the presence of a non-aqueous solvent, compared to the commonly applied literature values.
This allows a significant difference in the PG head group volume to be observed; having a larger volume than observed for the same system in water.
This suggests that the transfer of phospholipids to a DES is not just a simple substitution of the subphase.
In this specific case, we have proposed an explanation based on the dissociation of the PG head group salt and the subsequent interaction with the DES.

Finally, MCMC sampling was used to understand the inverse uncertainties present in the modelling parameters, enabling a confidence interval to be quoted alongside the most probable value.
The use of MCMC sampling also allowed the quantification of the correlations between the parameters of the chemically-consistent model.
This show the significant correlations between the head layer thickness and the volume fraction of solvent, and the head layer thickness and the tail layer thickness that becomes more prominent for short-tailed phospholipids.
The quantification of these correlations gives us a better understanding of the uncertainties on the parameters and can be rationalised based on the chemistry of the monolayers.
