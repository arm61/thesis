\section{Methods}

The computational methodology described herein has been implemented in the open-source C\_MPI program \texttt{fitoog} \cite{mccluskey_arm61/fitoog_2019}.

\subsection{Simulation Methodology}

The \texttt{fitoog} software takes as input a series of input files that define the molecules and the intramolecular interactions. 
The molecule input file is space separated consisting of an index, particle name, x-coordinate, y-coordinate, z-coordinate, and scattering length. 
While, the intramolecular interactions file describes the bonds within the molecule, in lieu of angular interactions a pseudo-bond is defined between each second particle in the system. 
When the initial molecule input file is read in, a differences object is created which stores the differential between the particle description and the more severe coarse-grained description that is used in the particle swarm optimisation. 

Inspired by the work on directional colloid self-assembly of Law \emph{et al.} \cite{}, a severe coarse-graining methodolgy was applied to the surfactant molecules.
This allowed for a significantly reduced parameter dimensionality to which the particle swarm optimisation was applied. 
The severe coarse-graining involved reduction the surfactant to a `director' description; where each surfactant molecule is defined by a position and a direction, shown pictorially in Figure~\ref{fig:director}.
This reduced the parameter dimensionality to just six variables per molecule; three of which described the position of the centre-of-mass position of the molecule ($a$, $b$, $c$) and three that describe the orientaion of the surfactant in space ($\phi$, $\omega$, $\kappa$).
These were the variables that were exposed to the particle swarm optimisation method outlined in Section~\ref{sec:partswarm}.
%
\begin{figure}
    \centering
    \includegraphics[width=0.8\textwidth]{smallangle/director}
    \caption{A graphical description of the severe coarse-graining applied to the MARTINI description of the \emph{n}-decyltrimethylammonium surfactant molecule for the use of the particle swarm algorithm.}
    \label{fig:director}
\end{figure}
%

Following each iteration of the particle swarm algorithm, the director description of the surfactant was expanded (using the differences object mentioned above) to a MARTINI representation (or whatever representation was used as an input).
The molecule was then rotated based on a rotation matrix, in this work the rotation matrix was constructed by first rotating the rotation axis by $-\phi$ and $-\omega$, then rotating by $\kappa$ in around the $z$-axis, before rotating the axis back to the original position by $\omega$ and $\phi$ \cite{evans_rotations_2001} (Figure~\ref{smallangle/rotdia} defines these angles).
With the severely coarse-grained description fully expanded, the system could be subjected to a potential model energy minimisation algorithm (implemented as a gradient descent) if desired before the scattering profile was calculated using the Golden Vectors method developed by Watson and Curtis \cite{watson_rapid_2013}.
The agreement between the calculated scattering profile and the `experimental' scattering was used as a figure of merit, $\zeta$, that was to be optimised by the particle swarm algorithm. 
%
\begin{figure}
    \centering
    \includegraphics[width=0.8\textwidth]{smallangle/rotdia}
    \caption{The definitation of the polar angles used in the coarse grained representation of the surfactant molecule.}
    \label{fig:rot}
\end{figure}
%

\subsection{Parallelisation}

The use of a population based optimisation methods allowed for easy access to highly parallel simulation. 
This was achieved but parallelisation over the cores available to the simulation such that a population is split evenly across them. 
There is intercore messaging that makes use of the MPI libraries, where the figure of merit from each core is shared with a parent core that determines the global best population. 
This global best is then shared to all cores of the process. 
Figure~\ref{fig:para} described the parallelisation method pictorially.
The parallelisation methodologies is to reduce the time spent when the helper cores are not in use. 
I feel that, while more sophisticated methodologies exist (such having a core-level best population and only occasionally finding the true global best), this implementation offers a useful test-case for assessing the utility of this optimisation methodology. 
%
\begin{figure}
    \centering
    %\includegraphics[width=0.8\textwidth]{smallangle/parallel}
    \caption{A schematic of the parallelisation methodology applied currently in the \texttt{fitoog} code.}
    \label{fig:para}
\end{figure}
%