\section*{Context}
The work contained in this chapter moves from coarse-grained representations of interfaces to those of solutions.
The use of classical simulation as a tool to assist small angle scattering analysis is popular in the biomolecular community \cite{perkins_atomistic_2016,hub_interpreting_2018}.
However, the use in the study of other soft matter species, such as micelles, is limited due to the significant computational cost of obtaining a starting structure that is similar to the experimental system.
While it is possible to produce a series of simulation of various micelle sizes and compare these with experimental data, the desire to model the inter-micelle interactions means that these method quickly become unfeasible.
Herein, the generation of a starting structure for an molecular dynamics simulation is treated as an optimisation problem through the application of a particle swarm optimisation method, where the optimisation target is an experimental scattering profile.
The aim is to obtain a starting structure that is representative of the experimental system, including scattering from inter-micelle interactions. 
