\section{Introduction}

Small angle scattering is a popular technique for the structural investigation of surfactant micelles \cite{sanchez-fernandez_micellization_2016}.
Typically, coarse shape-based modelling (such as that introduced in Section~\ref{sec:sasanal}) is used for the analysis.
These shapes allow for the classification of the micelle shape and interactions.
However, as with reflectometry, there has been a growing interest in the use of atomistic simulations as a multi-modal analysis tool for solution scattering methods \cite{ivanovic_temperature-dependent_2018}.

The use of atomistic simulation as an analysis method in the study of biomolecules has been popular for many years \cite{perkins_solution_1991,mayans_demonstration_1995,hub_interpreting_2018}.
These have built on the success of biomolecular simulation and crystallography, using structural information from the protein data bank \cite{noauthor_rcsb_nodate} and applying popular all-atom potential models.
Typically, this is used for the study of systems where the solution state differs significantly from that present in the crystal.
Popular systems that this analysis is applied to are flexible protein multimers, the benefit of molecular dynamics being that an ensemble of structures can be represented in a single simulation trajectory \cite{chen_validating_2014,bowerman_determining_2017}.

The importance of molecular dynamics simulation to represent an ensemble of structures has lead to the application of interesting aspect from probability theory.
In particular, the use of Bayesian inference has been popular in understanding that presence of different structures in solution.
For example, in the work of Bowerman \emph{et al.} \cite{bowerman_determining_2017}, accelerated molecular dynamics simulations (similar to traditional molecular dynamics however a `boost' potential is applied to the system to improve sampling) were performed on an all-atom representation of the protein multimer tri-ubiquitin.
The scattering profile was calculated from the simulation and evaluated against experimental data, and the agreement between the simulation and experiment assess in a Bayesian fashion, with a uniform prior probability.
This methodology showed that the presence of a two state ensemble was more probable that the single state that would be obtained from, for example, a crystallographic study.



MICELLE SIMULATION BY atomistic

MICELLE SIMULATION BY DPD

A major stumbling block of blah blah.

review of ivanovic paper.
