\section{Results \& Discussion}
The aim of this work was to produce a well-parallelised software capable of quickly producing starting structures for later MD simulations of multiple micellar species from SAS data.

\subsection{Parallelisation}
The parallelisation of a software package is commonly assessed using two metrics, strong and weak scaling.
These assess the CPU-bound\footnote{Central processing unit} efficiency and memory-bound efficiency of the software respectively.
A perfectly parallelised software would present a strong and weak scaling efficiency of 1 regardless of the number of processors.

In order to determine the strong scaling relationship for \texttt{fitoog}, a system was set up with a population size of 128 and was run for 5000 iterations.
This system was run on a range of processor counts, from 1 to 128,\footnote{Increasing in a $\log_2$ fashion.} on the SCARF cluster of STFC.
Figure~\ref{fig:scale}(a) shows the strong scaling relationship for \texttt{fitoog} running on up to 128 cores.
The weak scaling was probed by increasing the population size alongside the number of processors, both were varied in the same range as for the strong scaling, e.g. a population of 1 on a single core up to a population of 128 over 128 cores.
The weak scaling relationship is shown in Figure~\ref{fig:scale}(b).
%
\begin{figure}[t]
    \centering
    \includegraphics[width=\textwidth]{smallangle/scaling}
    \caption{The (a) strong and (b) weak scaling relationships of \texttt{fitoog} running on upto 128 cores of the SCARF cluster. The slight increase between \num{1} and \num{2} cores is most likely due to small difference in the speed of the different processors.}
    \label{fig:scale}
\end{figure}
%

It can be seen that both the strong and weak parallel efficiency of \texttt{fitoog} are relatively good, with the efficiency not dropping below \SI{80}{\percent} even when spread over 128 cores.
The speedup of \texttt{fitoog} is shown in Figure~\ref{fig:speedup},\footnote{This is generated from the strong scaling relationship.} from Amdahl's law\autocite{amdahl_validity_1967} it is possible to find that the parallel component of a given \texttt{fitoog} run makes up \SI{99.8}{\percent} of the computation.
This suggests that the parallelisation methodology currently implemented is successful, and it would not be advantageous to utilise a more sophisticated parallelisation method.
%
\begin{figure}[t]
    \forcerectofloat
    \centering
    \includegraphics[width=\textwidth]{smallangle/speedup}
    \caption{The speed up of \texttt{fitoog} running on upto 128 cores of the SCARF cluster, the blue dots show the speedup at different compute size, the orange line indicates the theoretical maximum, and the blue line shows the fitting Amdahl's law to the measured speedup.}
    \label{fig:speedup}
\end{figure}
%

The high efficiency in both strong and weak scaling regimes and the indication that the serial component of the computation is very small indicate that it is sensible to utilise high-performance computing resources for this software package.
In the real data application of this work, the \texttt{fitoog} software was run on 48 cores of the SCARF cluster.
This was chosen as it would spread the computation over exactly two nodes of the SCARF cluster,\footnote{Each node contains 24 cores.} therefore both nodes are being used to their full capacity.

\subsection{Test system}
In order to assess the PSO implementation, a simple test system was defined.
This consisted of a coordinate cell that contained four surfactant molecules at four corners of a \SI{20}{\angstrom} cube, each orientated in a different direction, see Figures~\ref{fig:test}(a-c).
The scattering intensity was calculated from the cell, with the blue beads given a scattering length of \SI{100}{\femto\meter} and the grey beads a scattering length of \SI{20}{\femto\meter}, to ensure the presence of intense scattering.
The scattering was calculated using the Debye equation\autocite{debye_zerstreuung_1915} for values of $q$ in a range from \SIrange{0.3}{1.5}{\per\angstrom} with 100 data points, this profile is shown in Figure~\ref{fig:test}(d).
%
\begin{figure}[t]
    \forceversofloat
    \centering
    \includegraphics[width=\textwidth]{smallangle/fake_box}
    \includegraphics[width=\textwidth]{smallangle/fake}
    \caption{Test system coordinate cell observed down the (a) \emph{x}-, (b) \emph{y}, and (c) \emph{z}-axis, and the calculated scattering data from the Debye equation.}
    \label{fig:test}
\end{figure}
%

\texttt{fitoog} was used to fit the ``experimental'' data; a population size of 100 was iterated over 5000 steps.
Ten repetitions of the \texttt{fitoog} run were performed,\footnote{The random seed and therefore the initial starting configuration varied between each run.} taking around two and a half minutes per run on a workstation computer with four cores.
Figure~\ref{fig:test_assess} shows the optimised scattering profile obtained from each of the runs and compares with the ``experimental'' data.
It is clear that some of the runs agree well with the data, in particular runs \num{1} and \num{2}, the resulting coordinate cell for these profiles are also shown in Figure~\ref{fig:test_assess}.
%
\begin{figure}
    \centering
    \includegraphics[width=0.49\textwidth]{smallangle/fake_assess1}
    \includegraphics[width=0.49\textwidth]{smallangle/fake_assess2} \\
    \includegraphics[width=0.49\textwidth]{smallangle/fake_assess3}
    \includegraphics[width=0.49\textwidth]{smallangle/fake_assess4} \\
    \includegraphics[width=0.49\textwidth]{smallangle/fake_assess5}
    \includegraphics[width=0.49\textwidth]{smallangle/fake_assess6} \\
    \includegraphics[width=0.49\textwidth]{smallangle/fake_assess7}
    \includegraphics[width=0.49\textwidth]{smallangle/fake_assess8} \\
    \includegraphics[width=0.49\textwidth]{smallangle/fake_assess9}
    \includegraphics[width=0.49\textwidth]{smallangle/fake_assess10} \\
    \includegraphics[width=\textwidth]{smallangle/fake_result}
    \caption{The best fit from the \texttt{fitoog} run (orange line) is compared with the ``real experimental'' data (blue line) for each of the ten runs. The result of runs 1 (a, b, and c) and 2 (d, e, and f) along each axis for the test system coordinated cell.}
    \label{fig:test_assess}
\end{figure}
%

This agreement with the ``experimental data'' in the test case is a positive result, and the resulting cells are shown in Figure~\ref{fig:test_assess} appear to show quantitative agreement\footnote{The molecules are in similar locations.} with the coordination cell from which the ``experimental'' data was found.
In particular, the agreement between the simulated and ``experimental'' data sufficient to consider that in the more disordered example it may be possible to form a realistic starting structure.
Therefore, the use of a PSO method was continued and applied to the real experimental data.

\subsection{Real data}
\label{sec:real_data}
This real experimental data consisted of a single SANS profile for the hydrogenated \emph{n}-decyltrimethylammonium micelle, with nitrate counter ions\footnote{Abbreviated to \ce{C_{10}TA+} and \ce{NO3-} respectively.} counter ions in \ce{D2O}.
It was assumed that this data was completely background subtracted such that the scattering present was a result of the micelles alone.
Figure~\ref{fig:expdata} shows the scattering profile that was being modelled.
%
\begin{figure}[t]
    \centering
    \includegraphics[width=\textwidth]{smallangle/exp_data}
    \caption{The experimental SANS data, from a solution of \ce{C_{10}TANO_3}, to which the real \texttt{fitoog} run was attempting to fit.}
    \label{fig:expdata}
\end{figure}
%

The aim of the application to real data was to attempt to quickly produce a system with multiple micellar species.
Therefore, the simulation cell was substantially larger; containing 500 MARTINI coarse-grained \ce{C_{10}TA+} and \ce{NO3-} molecules, the available cell was a cube with a side \SI{177}{\angstrom} in length.\footnote{No solvent was included in the box as the scattering was considered to have arisen from the micelle scattering alone. The concentration of the solution was \SI{\sim0.15}{\mol\deci\meter^{-3}}, which is nearly three times the cmc for \ce{C_{10}TA+} from \cite{rodriguez_surface_2007}.}
In order to assess the utility for the PSO, two \texttt{fitoog} runs were performed.\footnote{Each with five repetitions, and a population size of 96 over 500 iterations.}
The PSO method was compared with a random method, where at each iteration a new random population was generated.
Figure~\ref{fig:chi} shows the variation in the figure of merit, $\zeta$, in each of these optimisations.
Figure~\ref{fig:chi}(a-c) shows the best structure that was obtained from the \texttt{fitoog} runs, notably it is from the randomisation based run, which shows no evidence of the formation of micelle-like species.
From this, there is no clear benefit to the use of the PSO method over simply selecting random structures.
%
\begin{figure}
    \centering
    \includegraphics[width=\textwidth]{smallangle/comparechi}
    \includegraphics[width=\textwidth]{smallangle/best_real_bad}
    \caption{The quality of agreement between the \texttt{fitoog} model and the experimental data; where the blue lines are the different PSO runs and the orange are the randomisation runs. Result of the best outcome from the real data example, observed down the (a) \emph{x}-, (b) \emph{y}, and (c) \emph{z}-axis.}
    \label{fig:chi}
\end{figure}
%

This inability for the PSO method to optimise the structure of the micellar species could be due to a wide variety of reasons, some which could be acted on and others that could not.
While it is common to use values for the acceleration coefficients that are typically in the range of \numrange{0}{2}, it may be necessary to optimise these values.\footnote{Which, of course, could lead to an infinite set of optimisations.}
However, it may the case that the parameter space was too large to be optimised using the PSO alone, this is very likely considering that the dimensionality of the parameter space was 4500.\footnote{$500\times6+500\times3$, for 500 \ce{C_{10}TA+} and \ce{NO3-} molecules.}

A possible method that may enable the optimisation of such structures would be the inclusion of an energetic term.
For example, this could involve the use of an energy optimisation to be performed alongside the structural optimisation to the scattering profile, regardless of the optimisation methodology.
The use of energetic considerations and a Markov state model optimisation has been shown to perform well for peptide self-assembly.\autocite{sengupta_automated_2019}
Additionally, the Empirical Potential Structure Refinement used by Hargreaves \emph{et al.}\autocite{hargreaves_atomistic_2011} and in a coarse-grained fashion by Soper and Edler\autocite{soper_coarse-grained_2017} and the method used by Ivanovic \emph{et al.}\autocite{ivanovic_temperature-dependent_2018} all involve performing an energetic optimisation that is biased on the agreement with experimental data.
This use of an energetic consideration could be included in \texttt{fitoog} by performing an energy minimisation step following each step of the PSO.
This would have the added benefit of creating a range of surfactant conformations allowing a more realistic structure to form.
