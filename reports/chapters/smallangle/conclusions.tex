\section{Conclusions}
This chapter presented the development of a software entitled \texttt{fitoog}, the aim of which was to try and use PSO in order to generate a resonable starting structure for a MD simulation of multiple micelles in solution.
Previously work had shown it was possible to build a single micelle that agreed well with dilute experimental data.
However, there had been no previous work investigating the generation of multiple micelles based on the scattering data alone.

\texttt{fitoog} is a highly parallelised software, capable of running efficiently on high performance computing resources.
It was determined that the serial component of running \texttt{fitoog} made up \SI{0.2}{\percent} of the overall calculation.
From ten repetitions, the software was able to resolve the expected structure for a model test case quickly running on a workstation class machine.
However, when applied to a larger, realistic system it was not possible to obtain any realistic structures.
Additionally, when compared with a simple random number generation, it performed no better in minimisation of the figure of merit. 
This indicates that it may be necessary to consider energetic information in additional to structural detail to accurately develop a feasible starting structure for a MD simulation. 