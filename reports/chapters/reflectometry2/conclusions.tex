\section{Conclusions}

This chapter presents a direct comparison between a traditional method for the analysis of NR measurements with an analysis method using simulations from a range of all-atom and coarse-grained MD potential models.
It was shown that the MARTINI potential model could not accurately reproduce the experimental NR data, likely, due to the limitations of the 4-to-1 beading system when applied to a carbon chain of 18 atoms.

The Berger united atom and Slipids all-atom potential models both showed good agreement with the experimental data, however, the best agreement was obtained from the traditional chemically-consistent layer model.
This would be expected given that the chemically-consistent model contains many more ``degrees of freedom'' than the simulations which are severely chemically-constrained by the potential model.

Finally, some points from the highest resolution, Slipid, simulations were noted that may be used to improve the traditional monolayer model.
For example, it is desirable to model non-uniform solvation of the head group region which would enable a more accurate modelling of the lipid monolayer and the use of a conformal roughness may not be the best constraint to apply.
Application of these improvements may enable the more accurate modelling of phospholipid monolayers from NR.
