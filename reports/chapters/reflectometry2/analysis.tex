\section{Data analysis}
\subsection{Traditional layer-model analysis}
In order to provide a point of comparison for the simulation-derived methods, the chemically-consistent reflectometry model developed in Chapter~\ref{reflectometry1} was used for the analysis of the experimental data.
The only modification that was made to the methodology described in Chapter~\ref{reflectometry1} was that the volume of the phospholipid tail group, $V_t$ was constrained based on the APM (taken from the surface pressure-isotherm data),
%
\begin{equation}
V_t = d_t\text{APM},
\end{equation}
%
where, $d_t$ is the tail layer thickness. The result of this constraint is that both the monolayer model and the simulation-derived models were constrained equally by this measured surface coverage.

\subsection{Simulation-dervied analysis}
A custom-class, \texttt{md\_simulation}, was developed for \texttt{refnx} \cite{nelson_refnx_2019,nelson_refnx_2019-1} at enabled the determination of a reflectometry profile from simulation, using a similar method to that employ in previous work, such as Dabkowska \emph{et al.} \cite{dabkowska_modulation_2014}.
The Abel\`{e}s layer model formalism is applied to layers, the SLD of which is drawn directly from the simulation, and the thickness of which is defined.
The layer thickness used was \SI{1}{\angstrom} for the Slipid and Berger potential model simulations, with an interfacial roughness between these layers of \SI{0}{\angstrom}.
For the MARTINI potential model, a layer thickness of \SI{4}{\angstrom} was used, with an interfacial roughness of \SI{0.2}{\angstrom}. Each of the \SI{50}{\nano\second} production simulations were analysed each \SI{0.1}{\nano\second}, and the SLD profiles were determined by summing the scattering lengths, $b_j$, for each fo the atoms in a given layer.
%
\begin{equation}
\text{SLD}_n = \frac{\sum_j b_j}{V_n},
\end{equation}
%
where, $V_n$ is the volume of the layer $n$, obtains from the simulation cell parameters in the plane of the interface and the defined layer thickness.
A uniform background was assigned based on the intensity at the highest $q$-vector and scale factor were then determined using \texttt{refnx} to offer the best agreement between the calculated reflectometry profile and that measured experimentally.

\subsection{Comparison between monolayer model and simulation-derived analysis}
In order to assess the agreement between the model from each method, the following goodness-of-fit metric was used, following the transformation of the data into $Rq^4$ space,
%
\begin{equation}
\chi^2 = \sum_{i=1}^{N_{\text{data}}}{\frac{[R_{\text{exp}}(q_i) - R_{\text{sim}}(q_i)]^2}{[\delta R_{\text{exp}}(q_i)]^2}},
\end{equation}
%
where, $q_i$ is a given $q$-vector, $R_{\text{exp}}(q_i)$ is the experimental reflected intensity, $R_{\text{sim}}(q_i)$ is the simulation-derived/traditionally-developed reflected intensity, and $\delta R_{\text{exp}}(q_i)$ is the resolution function of the experimental data.

Parametric outcomes from the different analysis methods were also compared, such as the number of water molecules per head group, wph.
This was obtained from the monolayer model by considering the solvent fraction in the head-layer, $\phi_h$, the volume of the head group, $V_h$, and taking the volume of a single water molecule to be \SI{29.9}{\angstrom\cubed} (from the density of water as \SI{997}{\kg\per\meter\cubed}),
%
\begin{equation}
\text{wph}=\frac{\phi_h V_h}{29.9 - 29.9\phi_h}.
\label{equ:wph}
\end{equation}
%
In MD simulations, the number density for each component in the system are obtained directly from the trajectory.
The ratio of the lipid heads and the water then give the wph.
The range for this was taken as between the \SI{20}{\percent} and \SI{80}{\percent} quantiles of the lipid head layers.
Another parameter that was considered from the MD simulations was the length of the carbon tail in the molecules.
This was determined as the length of a vector from the first atom, or bead, of the tail group to the last atom.
This was calculated for each carbon tail in each molecule, across all of the timesteps analysed, and are quoted as the median values with a \SI{95}{\percent} quantile of the underlying sample. 
