\section*{Abstract}
The use of molecular simulation to aid in the analysis of neutron reflectometry measurements is commonplace.
However, reflectometry is a tool to probe large-scale structures, and therefore the use of all-atom simulation may be irrelevant.
This work presents the first direct comparison between the reflectometry profiles obtained from different all-atom and coarse-grained molecular dynamics simulations and the reflectometry profiles from a chemically-consistent layer modelling method.
We find that systematic limitations reduce the efficacy of the MARTINI potential model, while the Berger united-atom and Slipids all-atom potential models agree similarly well with the experimental data.
The chemically-consistent layer model gives the best agreement, however, the higher resolution simulation-dependent methods produce an agreement that is comparable.
