\chapter{Additional Code Blocks for Differential Evolution} % Main appendix title

\label{diff_evo_app} % Change X to a consecutive letter; for referencing this appendix elsewhere, use \ref{AppendixX}

This appendix includes additional Code Blocks relevant to the introduction of the DE optimisation method discussed in Section~\ref{sec:de}.
Code Block~\ref{cb:mut} describes the mutation step of the DE algorithm, as described by Bj\"{o}rck.\autocite{bjorck_fitting_2011}
%
\begin{listing}
    \centering
    \caption{The mutation step used in a classical trial method for a DE algorithm, as described in \cite{bjorck_fitting_2011}. The input variables are \texttt{p} which is an array of floats representing the parent population, \texttt{b} which is an array of floats representing the best vector of the parent population, and \texttt{km} the mutant constant. This function returns an array of floats representing the mutant vector.}
    \lstinputlisting{reports/code_blocks/mutation.py}
    \label{cb:mut}
\end{listing}
%
Code Block~\ref{cb:recomb} describes the recombination step of the DE algorithm, as described by Bj\"{o}rck.\autocite{bjorck_fitting_2011}
%
\begin{listing}
    \centering
    \caption{The recombination step used in a classical trial method for a DE algorithm, as described in \cite{bjorck_fitting_2011}. The input variables are \texttt{p} which is an array of floats representing the parent population, \texttt{m} which is an array of floats representing the mutant vector, and \texttt{kr} the recombination constant. This function returns an array of floats representing the offspring population.}
    \lstinputlisting{reports/code_blocks/recombination.py}
    \label{cb:recomb}
\end{listing}
%
Code Block~\ref{cb:self} describes the classical step of the DE algorithm, as described by Bj\"{o}rck.\autocite{bjorck_fitting_2011}
%
\begin{listing}
    \centering
    \caption{The classical selection step used in a DE algorithm, as described in \cite{bjorck_fitting_2011}. The input variables are \texttt{p} which is an array of floats representing the parent population, \texttt{m} which is an array of floats representing the offspring vector, and \texttt{f} is the function to be minimised. This function returns an array of floats representing the new parent population.}
    \lstinputlisting{reports/code_blocks/selection.py}
    \label{cb:sel}
\end{listing}
%
